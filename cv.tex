\documentclass[11pt,a4paper,nolmodern]{moderncv}
\usepackage[noindent,UTF8]{ctex}
\usepackage{info}

\title{清华大学软件学院}
\myquote{自强不息,厚德载物。}{}

\begin{document}
\setmainfont{Minion Pro}
\setsansfont{Myriad Pro}

\hyphenpenalty=10000
\maketitle

\section{教育背景}
\tlcventry{2009}{0}{清华大学(THU),软件学院(SS),本科在读}{}{}{}{
\begin{itemize}
  \item 上一学期排名:1/54,GPA:94.9/100
  \item 全部学期排名:3/54,GPA:90.5/100
  \item 2011年,清华大学综合优秀奖学金
  \item 2010年,清华大学综合优秀奖学金
\end{itemize}}
\vspace{0.5em}
\tlcventry{2006}{2009}{甘肃省兰州第一中学}{}{}{}{
\begin{itemize}
  \item 2008年全国高中数学联赛一等奖
  \item 2008年全国青少年信息学奥林匹克联赛一等奖
  \item 2007年全国青少年信息学奥林匹克联赛一等奖
\end{itemize}}

\section{经历}
\subsection{项目经历}
\tlcventry{2011}{0}{人人网应用——清华晒课厅}{}{}{}{
\begin{itemize}
 \item 使用ASP .Net + SQL Server开发
 \item 应用安装量超过10000
 \item 第二版(\href{https://github.com/terro/shaike}{GitHub链接})正在开发
  \item 受创新工厂与清华大学合作的创新计划支持
\end{itemize}}
\vspace{0.5em}
\tldatelabelcventry{2011}{2011秋}{简单Python程序到C\#的翻译器——Py\#}{}{}{}{
\begin{itemize}
  \item 使用Lex和Yacc作为词法分析和语法分析工具
  \item \href{https://github.com/terro/Py-Sharp}{GitHub链接}
\end{itemize}}
\vspace{0.5em}
\tldatelabelcventry{2011}{2011春}{基于Xilinx FPGA的片上系统开发}{}{}{}{
\begin{itemize}
  \item 使用VHDL及C在Spartan-II 300E LC上进行开发
  \item 完成了基于串口通信及LCD屏幕的猜拳游戏
\end{itemize}}
\vspace{0.5em}
\tldatelabelcventry{2011}{2011春}{基于TI单片机的简单系统开发}{}{}{}{
\begin{itemize}
  \item 使用TI MSP430单片机进行开发
  \item 完成了基于超声波测距模块的距离和速度测量装置
\end{itemize}}
\vspace{0.5em}
\tlcventry{2009}{0}{其他课程项目、研究}{}{}{}{
\begin{itemize}
 \item 在教学用操作系统xv6中使用FAT32替换了原有的简单文件系统
 \item 完成了基于Qt4的嵌入式MP3播放器的设计与实现(\href{https://github.com/terro/A-Simple-Embedded-MP3-Player-Based-On-ARM-9}{GitHub链接})
 \item 完成了具有加密、硬件信息测试功能的MBR开发
 \item 阅读了HyperSQL数据库源代码,并从存储、查询、事务、并发和恢复等方面对代码进行了分析
\end{itemize}}

\subsection{社会工作}
\tlcventry{2011}{0}{清华大学软件学院学生科协副主席}{}{}{}{
\begin{itemize}
 \item 主管软件学院科协赛事部
\end{itemize}}
\vspace{0.5em}
\tlcventry{2011}{0}{清华大学勤工助学大队信息系统服务部部长}{}{}{}{
\begin{itemize}
 \item 协调部门内各项目组和队员的工作
\end{itemize}}
\vspace{0.5em}
\tlcventry{2010}{2011}{清华大学软件学院学生科协赛事部部长}{}{}{}{
\begin{itemize}
 \item 北京市高校软件设计邀请赛主办方负责人
 \item 组织、主办大一新生算法班
 \item 获清华大学学生科协优秀学生干部称号
\end{itemize}}

\subsection{公益活动}
\tlcventry{2009}{0}{首都机场志愿者、中国科技馆志愿者、中网志愿者}{}{}{}{
\begin{itemize}
 \item 累计完成志愿服务89小时
 \item 获清华大学三星志愿者称号
\end{itemize}}
\vspace{0.5em}
\tldatelabelcventry{2010}{2010夏}{中美联合支教赴青海乐都支教活动}{}{}{}{
\begin{itemize}
 \item 在青海乐都高级实验中学支教10天
\end{itemize}}

\section{能力}
\subsection{开发}
\cvcomputer{语言}{C/C++, C\#, Python, Java}
           {Web}{HTML, ASP .Net, jQuery, JavaScript, AJAX}
\cvcomputer{框架}{Django}
           {数据库}{MySQL, SQL Server, MongoDB, Redis}
\cvcomputer{代码管理}{SVN, Git}
           {工具}{Redmine, GitHub}

\subsection{其他}
\cvcomputer{Office}{OpenOffice/LibreOffice, Microsoft Office}
           {操作系统}{GNU/Linux(Ubuntu, Mint), Windows}
\cvcomputer{排版}{\XeLaTeX{}}
           {编辑器}{VIM, sublime text}

\section{语言}
\cvlanguage{汉语}{母语}{}
\cvlanguage{英语}{熟练}{通过全国大学英语四级考试,通过清华大学英语水平一考试}

\section{个人兴趣}

\cvhobby{运动}{游泳}
\cvhobby{互联网}{\href{https://twitter.com/terro1991}{Twitter}}
\cvhobby{}{\href{http://facebook.com/dangfan}{facebook}}
\cvhobby{}{\href{http://github.com/terro}{GitHub}}
\cvhobby{}{stackoverflow}
\cvhobby{其他}{旅游,读书,摄影}

\end{document}

